\documentclass{article}
\usepackage{graphicx,titling}
\usepackage{fancyhdr}
\usepackage[backend=biber]{biblatex-chicago}
\addbibresource{ref.bib}
\usepackage{hyperref}

\DeclareBibliographyDriver{archiveitem}{%
  \printfield{title}
  \newunit\newblock
  \printlist{author}%
  \newunit\newblock
  \printfield{reel}%
  \newunit\newblock
  \printfield{frame}%
  \newunit\newblock
  \printfield{note}%
  \finentry
}

\DeclareBibliographyAlias{archiveitem}{misc}

\usepackage[a4paper, headheight=1in, margin=1in]{geometry}

\fancyhf{}
\fancyhead[c]{\centering{\includegraphics[width=0.05\textwidth]{logo.png}}\\ \textsc{University of Indiana}}
\fancyhead[r]{Keesbury \thepage}

\newcommand{\logo}[2]{
  \pretitle{
  \begin{center}\includegraphics[width=0.2\textwidth]{#1}\\
  \large{\textsc{#2}}\Large
  \vskip10pt}
  \posttitle{\end{center}}
}

\logo{iu_trident_web_crimson.png}{University of Indiana}
\title{Notes on Military-Carceral, and Military-Clinical Spaces of the American Civil War}
\author{Autumn Keesbury}
\date{Fall 2025}

\begin{document}
  \maketitle
  \thispagestyle{empty}
  \newpage
  \pagestyle{fancy}

  \section*{Military-Carceral Spaces}
  During the Civil War, we see many instances of the intersection of the carceral and the military: of course in the form of the Prisoner-Of-War (POW) camp,
  but also in the less obvious forms of the internal disciplinary structures of the military, the observation and documentary knowledge of troops, and, more
  abstractly, in the panoptic geometries of both POW and military camps. I wish to explore these points of intersection, and in particular I wish to do this
  with reference to the particular convergences and divergences between the carceral spaces necessitated and developed by military means, and those more
  classical carceral spaces of the type Foucault investigated in \textit{Discipline and Punish} \autocite{Foucault1995}.

  \section*{Military-Clinical Spaces}
  The approach I am taking to the intersection of the military and the carceral during the Civil War has been studied in some depth by \citeauthor{McNutt2024}
  and several others \autocites{McNutt2024}{McNutt2021}{McNutt2019}{McNutt2019a}. McNutt makes frequent reference to Foucault's \textit{Discipline and Punish} as a 
  means of understanding the relations of power in Civil War military prisons, and indeed the Foucaultian analysis of spaces is somewhat commonplace in military-geographic studies on POW camps (see, for instance, \citeauthor{Moran2022}). The same cannot be said of the approach I wish to take in understanding the
  intersection of the military and the clinical.

  Indeed, much of the literature on the medical aspects of the Civil War concerns the techniques of care, or otherwise the more human aspects 
  (\citeauthor{Devine2016} \citedate{Devine2016}).

  \newpage
  \fancyhead[c]{References}
  \nocite{*}
  \printbibliography
\end{document}

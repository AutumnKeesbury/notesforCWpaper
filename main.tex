\documentclass{report}
\usepackage{graphicx,titling,titlesec, csquotes}
\usepackage{fancyhdr}
\usepackage[backend=biber]{biblatex-chicago}
\addbibresource{ref.bib}
\usepackage{hyperref}

\DeclareBibliographyDriver{archiveitem}{%
  \printfield{title}
  \newunit\newblock
  \printlist{author}%
  \newunit\newblock
  \printfield{reel}%
  \newunit\newblock
  \printfield{frame}%
  \newunit\newblock
  \printfield{note}%
  \finentry
}

\titleformat{\paragraph}{\normalfont\normalsize\scshape}{}{1em}{}

\DeclareBibliographyAlias{archiveitem}{misc}

\usepackage[a4paper, headheight=1in, margin=1in]{geometry}

\fancyhf{}
\fancyhead[c]{\centering{\includegraphics[width=0.05\textwidth]{logo.png}}\\ \textsc{University of Indiana}}
\fancyhead[r]{Keesbury \thepage}

\newcommand{\logo}[2]{
  \pretitle{
  \begin{center}\includegraphics[width=0.2\textwidth]{#1}\\
  \large{\textsc{#2}}\Large
  \vskip10pt}
  \posttitle{\end{center}}
}

\logo{iu_trident_web_crimson.png}{University of Indiana}
\title{Notes on Military-Carceral, and Military-Clinical Spaces of the American Civil War}
\author{Autumn Keesbury}
\date{Fall 2025}

\begin{document}
  \maketitle
  \thispagestyle{empty}
  \newpage
  \pagestyle{fancy}


  \chapter{Military-Carceral Spaces}\label{chap: 1} % (fold)
  
  \section*{Overview}
  During the Civil War, we see many instances of the intersection of the carceral and the military: of course in the form of the Prisoner-Of-War (POW) camp,
  but also in the less obvious forms of the internal disciplinary structures of the military, the observation and documentary knowledge of troops, and, more
  abstractly, in the panoptic geometries of both POW and military camps. I wish to explore these points of intersection, and in particular I wish to do this
  with reference to the particular convergences and divergences between the carceral spaces necessitated and developed by military means, and those more
  classical carceral spaces of the type Foucault investigated in \textit{Discipline and Punish} \autocite{Foucault1995}.

  % chapter  (end)

  \chapter{Military-Clinical Spaces}\label{chap: 2} % (fold)
  
  The approach I am taking to the intersection of the military and the carceral during the Civil War has been studied in some depth by \citeauthor{McNutt2024}
  and several others \autocites{McNutt2024}{McNutt2021}{McNutt2019}{McNutt2019a}. McNutt makes frequent reference to Foucault's \textit{Discipline and Punish} as a 
  means of understanding the relations of power in Civil War military prisons, and indeed the Foucaultian analysis of spaces is somewhat commonplace in military-geographic studies on POW camps (see, for instance, \citeauthor{Moran2022}). The same cannot be said of the approach I wish to take in understanding the
  intersection of the military and the clinical.

  Indeed, much of the literature on the medical aspects of the Civil War concerns the techniques of care, or otherwise the more individual aspects 
  (\citeauthor{Devine2016} \citedate{Devine2016}). It is this gap, most of all, that I wish to fill. In order to do this, I want to look at the notes of and
  correspondences of doctors working with the United States Sanitary Commission (USSC) both during and immediately following the war. In these materials 
  (those which are most prescient to this investigation are included in the References), I have noticed several very interesting patterns, and, since I have
  been reading these records at the same time as I have been reading Foucault's \textit{The Birth of the Clinic}, I have been able to connect theory to
  content, and \textit{vice versa} \autocite{Foucault1994}.

  These connections and patterns, and their meaning in the history of American medical thought, are as follows:
  \paragraph{Epidemic medicine}\label{par:epidemic_medicine} % (fold)
  I believe that much of the medicine of the Civil War constituted an epidemic medicine, in the sense of the 19th century medical understanding of the term.
  As Foucault describes, around the period we are considering, an epidemic was \textquote{more than a particular form of a disease ... it was an 
  autonomous, coherent, and adequate evaluation of disease} \autocite[pp. 23]{Foucault1994}. The epidemic is thus not described solely in relation to its
  effect on the patient, but as a sum of circumstances which was at once exactly equal to and much greater than its parts. Indeed, no possible factor,
  no matter how small or ordinary, was disregarded in the act of classification and negation of the epidemic. What is considered, too, is not merely a set
  of extrinsic or circumstantial conditions (climate, hygiene, etc.), but also includes intrinsic factors such as time of year, geographical location,
  and proximity to bodies of water. 

  As might be expected, such a phenomenon, understood as an integration of all possible variables over all others, poses a unique challenge in terms of medical
  response. It is for this reason that a medicine of epidemics demands a constant supervision and documentation, and that, through continuous a observation
  of all factors and their necessary interpretation and re-interpretation, the medicine of epidemics \textquote{circumscribes, where gazes meet, the
  individual, unique nucleus of these collective phenomena} \autocite[pp. 25]{Foucault1994}.
  % paragraph Epidemic medicine (end)
  % chapter  (end)

  \newpage
  \fancyhead[c]{References}
  \nocite{*}
  \printbibliography
\end{document}

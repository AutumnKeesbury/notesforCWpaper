\documentclass{report}
\usepackage{graphicx,titling,titlesec, csquotes,setspace}
\usepackage{fancyhdr}
\usepackage[backend=biber]{biblatex-chicago}
\addbibresource{ref.bib}
\usepackage{hyperref}
\usepackage[Sonny]{fncychap}

\linespread{1.15}

\titleformat{\paragraph}{\normalfont\normalsize\scshape}{}{1em}{}

\setlength{\skip\footins}{15pt}

\usepackage[letterpaper, headheight=1in, margin=1in]{geometry}

\fancyhf{}
\fancyhead[c]{\centering{\includegraphics[width=0.05\textwidth]{logo.png}}\\ \textsc{University of Indiana}}
\fancyhead[r]{Keesbury}
\fancyfoot[c]{\thepage}

\newcommand{\logo}[2]{
  \pretitle{
  \begin{center}\includegraphics[width=0.2\textwidth]{#1}\\
  \large{\textsc{#2}}\Large
  \vskip10pt}
  \posttitle{\end{center}}
}

\logo{iu_trident_web_crimson.png}{University of Indiana}
\title{Notes on Military-Clinical Spaces of the American Civil War}
\author{Autumn Keesbury}
\date{Fall 2025}

\begin{document}
  \maketitle

  \newpage
  \renewcommand\thepage{\romannumeral\numexpr\value{page}-1\relax}
  \tableofcontents
  \newpage

  \pagenumbering{arabic}

  \pagestyle{fancy}

  \chapter{An Epidemic Medicine}\label{chap: epidemic} % (fold)
  
  The approach I am taking to the intersection of the military and the carceral during the Civil War has been studied in some depth by \citeauthor{McNutt2024}
  and several others \autocites{McNutt2024}{McNutt2021}{McNutt2019}{McNutt2019a}. McNutt makes frequent reference to Foucault's \textit{Discipline and Punish} as a 
  means of understanding the relations of power in Civil War military prisons, and indeed the Foucaultian analysis of spaces is somewhat commonplace in military-geographic studies on POW camps (see, for instance, \citeauthor{Moran2022}). The same cannot be said of the approach I wish to take in understanding the
  intersection of the military and the clinical.

  Indeed, much of the literature on the medical aspects of the Civil War concerns the techniques of care, or otherwise the more individual aspects 
  (\citeauthor{Devine2016} \citedate{Devine2016}). It is this gap, most of all, that I wish to fill. In order to do this, I want to look at the notes of and
  correspondences of doctors working with the United States Sanitary Commission (USSC) both during and immediately following the war. In these materials 
  (those which are most prescient to this investigation are included in the bibliography), I have noticed several very interesting patterns, and, since I have
  been reading these records at the same time as I have been reading Foucault's \textit{The Birth of the Clinic}, I have been able to connect theory to
  content, and \textit{vice versa} \autocite{Foucault1994}.

  \section{Epidemic}\label{sec:epidemic} % (fold)
  I believe that much of the medicine of the Civil War constituted an epidemic medicine, in the sense of the 19th century medical understanding of the term.
  As Foucault describes, around the period we are considering, an epidemic was \textquote{more than a particular form of a disease ... it was an 
  autonomous, coherent, and adequate evaluation of disease} \autocite[pp. 23]{Foucault1994}. The epidemic is thus described not solely in relation to its
  effect on the patient, but as a sum of circumstances at once exactly equal to and much greater than its parts. Indeed, no factor,
  no matter how small or ordinary, was disregarded in the classification of, and response to, the epidemic. What is considered, further, is not merely a set
  of extrinsic or circumstantial conditions (climate, hygiene, etc.), but also includes intrinsic factors such as time of year, geographical location,
  and proximity to bodies of water. 

  Such a phenomenon, as might be expected, when understood as an integration of all possible variables over all others, poses a unique challenge in terms of medical
  response. It is for this reason that a medicine of epidemics demands a constant supervision and documentation, and that, through continuous observation
  of all factors and a ceaseless interpretation and re-interpretation of data, the medicine of epidemics \textquote{circumscribes, where gazes meet, the
  individual, unique nucleus of these collective phenomena} \autocite[pp. 25]{Foucault1994}. The knowledge of an epidemic, then, requires a complete
  yet contradictory set of facts, derived from numerous and heterogeneous gazes distributed regularly throughout the social body. 

  \section{Medical approach to epidemics}\label{sec:medical-epidemic}
  An organizational structure which facilitated the synthesis of a homogeneous picture of an epidemic through superimposing and cross-checking 
  medical gazes was present in the Armies of the United States during the American Civil War, at all levels of administrative functioning. Regimental
  surgeons and hospitals were tasked with documenting individual cases, meteorological data, and various other information situated at a similar
  level. Then there were the surgeons of the general hospitals, tasked with investigating and recording extraordinary cases which may be exemplary,
  but also with conducting research into the nature of various conditions which were difficult or time-consuming to treat. Lastly, and situated near
  the highest administrative level, were the doctors who worked directly for the USSC, and whose duty it was to visit and write high-level
  reports on the conditions of hospitals under the Sanitary Commission's control. Taken together, the USSC encompasses four parallel,
  unlimited series which extend the space of medical knowledge infinitely\autocite[pp.25]{Foucault1994}: the study of topographies 
  (conducted at the top level by doctors like S.B. Hunt\autocite{ussc:6:882}), meteorological observations (like those collected by Lyman 
  \autocite{ussc:6:775}), monitoring epidemics (see the reports on outbreaks in hospitals\autocite{ussc:9:1367}), and the description of extraordinary 
  cases (e.g. Howard's report on a case of death during the administration of chloroform\autocite{ussc:9:1340}).

  It is these parallel gazes which, through their integration into the military-political structures of the Armies of the United States, were able to cover
  a domain which was in many respects broader than that of the institutions to which they were formally subordinate. Indeed, everywhere the Armies
  of the Union went, they were followed by, or moved in lockstep with, the ever-vigilant gaze of a medicine of epidemics. Furthermore,
  there is often not just an isomorphism between the realized structures of the military and medical, but a whole series of such correspondences
  between their possible configurations: that is to say, a reorganization of one institution is accompanied by a similar reorganization of the
  other. In this sense, the medical gaze which \textit{knows} an epidemic forms a negative copy of the individualizing disciplinary gaze; what the
  positively-determined structures of the military do not cover, the medical gaze steps in to observe. Through the union of these two institutions,
  a complete and totally penetrating gaze is inscribed upon a space which is necessarily opposed, at each point, to their very workings.

  That this space is in opposition to the disciplinary techniques and careful ordering demanded by the military is well known. At any point
  in history, when a large mass of men are gathered together and marched into a territory where the typical normative institutions
  of society are weakened, there will be a problem of how to keep them restrained: to make their violence limited and useful; to render their
  bodies docile. I argue, however, that the introduction of individualizing military discipline is not alone able to impose an ordering upon
  this space -- or at least that this was the case as the scale and scope of conflict waxed towards the mid-nineteenth century.

  \chapter{The Medical Gaze}\label{chap: gaze} % (fold)

  In the previous chapter, I discussed the co-extension of the medical gaze and military discipline. The specific relationship of the doctor, the
  patient, and the disease is one which is crucial to any understanding of the medicine of the period. In this section, I wish to explore this relationship
  with reference to the specific context of the hospital camp and general hospital environments. Moreover, I will seek to draw direct links between
  works of theory and primary-source documents. 

  \vspace{12pt}
  
  \section{The general hospital}\label{sec:general hospital} % (fold)


  % section hospital (end)

  \printbibliography
  \addcontentsline{toc}{chapter}{Bibliography}
 
  \end{document}
